\documentclass{article}

\usepackage{amsmath}
\usepackage{amsfonts}
\usepackage{amssymb}
\usepackage{titlesec}
\usepackage{authblk}

\title{18 fantastici e dove trovarli}
\author{Sto Cazzo}
\affil{Dipartimento di Matematica, Università della Vita}

\begin{document}
\maketitle
\newpage

\begin{flushright}
  \hspace{-2cm}
  \textsc{C'ha la mamma puttana} \\
  \textit{Rodolfo ``Rodman'' Boccia}, de Poggilinis
\end{flushright}

\begin{flushright}
  \hspace{-2cm}
  \textsc{Sembra uno di quei documenti tristi da cui studi} \\
  \textit{Mia sorella}, prima reazione a questo pdf
\end{flushright}
\tableofcontents
\newpage

\section{coefficienti binomiali}
\begin{itemize}
\item \textbf{Formula:}
  \[ \binom{n}{k} = \frac{n!}{k!(n-k)!} \]

\item \textbf{Casi particolari:}
  \begin{align*}
    \binom{n}{k} &= \binom{n}{n-k} \\
    \binom{n}{n} &= \binom{n}{0} = 1 \\
    \binom{n}{n-1} &= \binom{n}{1} = n
  \end{align*}
\end{itemize}

\section{condizionate}
\begin{itemize}
\item \textbf{Formula}:
  mettiemo $A$ e $B$ due eventi
  \[ P(A|B) = \frac{P(A \cap B)}{P(B)} \]
\item \textbf{Bayes}:
  \[ P(B|A) = P(A|B) \frac{P(B)}{P(A)} \]
\item \textbf{L'altro}:
  \[ P(A,B) = P(A|B) P(B) \]
\item \textbf{Con cui marginalizza così}: (quando fai le bernoulli condizionate)
  \[ P(A) = \sum_b P(A,B = b) = \sum_b P(A|B = b) P(B = b) \]
\end{itemize}

\section{distribuzioni}

\subsection{discrete}
descritte da $\mathbb{P}(X = k)$

\begin{itemize}
  \item Bernoulli, lancio una moneta, 1 se testa, 0 se croce
    \begin{itemize}
    \item \textbf{Parametri}: $p$, la probabilità che esce testa
    \item \textbf{Formula}:
      \begin{align*}
        \mathbb{P}(X = 1) &= p \\
        \mathbb{P}(X = 0) &= 1-p
      \end{align*}
    \end{itemize}

  \item Binomiale, lancio $n$ bernoulli \emph{i.i.d.} con lo stesso parametro, conta le teste
    \begin{itemize}
    \item \textbf{Parametri}: $n$, il numero di bernoulli che lancio; $p$, il parametro delle bernoulli
    \item \textbf{Formula}:
      \[ \mathbb{P}(X = k) = \binom{n}{k} p^k {(1-p)}^{n-k} \]
    \end{itemize}

  \item Geometrica, lancio bernoulli costantemente finchè una non fa $1$, la prima era la k-esima per quale k?
    \begin{itemize}
    \item \textbf{Parametri}: $p$, il parametro della bernoulli sottostante
    \item \textbf{Formula}:
      \[ \mathbb{P}(X = k) = p{(1-p)}^{k-1} \]
    \end{itemize}

  \item Geometrica Modificata, stessa cosa della geometrica ma parto da 0 e non da 1
    \begin{itemize}
    \item \textbf{Parametri}: $p$, il parametro della bernoulli sottostante
    \item \textbf{Formula}:
      \[ \mathbb{P}(X = k) = p {(1-p)}^k \]
    \end{itemize}

  \item Poisson
    \begin{itemize}
    \item \textbf{Parametri}: $\lambda$ (tasso\footnote{wikipedia ipse dixit})
    \item \textbf{Formula}:
      \[ \mathbb{P}(X = k) = e^{-\lambda} \frac{\lambda^k}{k!} \]
    \end{itemize}
\end{itemize}

\subsection{continue}
descritte da
\begin{itemize}
\item \textbf{Legge}: $F_X (t) := \mathbb{P}(X \leq t)$
\item \textbf{Densità}: $f_X (t) := \frac{d}{dt}F_X(t)$
\end{itemize}

di queste abbiamo
\begin{itemize}
\item Uniforme
  \begin{itemize}
  \item \textbf{Parametri}: $a$, inizio intervallo; $b$, fine intervallo
  \item \textbf{Densità}:
    \[ f_X(t) = \begin{cases}
      \frac{1}{b-a} &\text{ se } t \in (a,b) \\
      0 &\text{ altrimenti }
    \end{cases}
    \]
  \item \textbf{Legge}:
    \[ F_X(t) = \begin{cases}
      0 & \text{ se } t \leq a \\
      \frac{t-a}{b-a} & \text{ se } t \in (a, b) \\
      1 & \text{ se } t \geq b
    \end{cases}
    \]
  \end{itemize}

\item Esponenziale
  \begin{itemize}
  \item \textbf{Parametri}: $\lambda$, la lambda
  \item \textbf{Densità}:
    \[ f_X(t) = e^{-\lambda x} \]
  \item \textbf{Legge}:
    \[ F_X(t) = 1 - e^{-\lambda x}\]
  \end{itemize}

\item Gaussiana
  \begin{itemize}
  \item \textbf{Parametri}: $\mu$, la media; $\sigma^2$, la varianza
  \item \textbf{Densità}: \\
    \[
    f_X(t) = \frac{1}{\sigma \sqrt{2 \pi}} e^{-\frac{1}{2} {(\frac{\mu - t}{\sigma})}^2}
    \]
    che nel caso standard ($\sigma = 1, \mu = 0$), fa
    \[
    f(t) = \frac{1}{\sqrt{\pi}} e^{-\frac{t^2}{2}}
    \]
  \item \textbf{Legge}: Lasciata come esercizio al lettore
  \end{itemize}

\item varie stronzate della Gaussiana includono
  \begin{itemize}
  \item Gaussiana non standard in funzione di $N(0,1)$ (traslazione affine)
    \[
    \alpha N(0,1) + \beta = N(\beta, \alpha^2)
    \]
  \item Quella standard ($N(0,1)$) è pari, quindi, detta $\Phi$, la sua distribuzione/legge
    \[
    \Phi(-x) = 1-\Phi(x) \text{ (questa è ovunque)}
    \]
  \end{itemize}
\end{itemize}

\subsection{modifiche di va}
\begin{itemize}
\item Affine \textit{(el Formulino)} \\
  \begin{align*}
    Y &= \alpha X + \beta \\
    f_Y &= \frac{1}{\lvert \alpha \rvert} f_x (\frac{x - \beta}{\alpha})
  \end{align*}
\item Quadrato \textit{(el Formulero)} \\
  \begin{align*}
    Y &= X^2 \\
    f_Y &= \frac{1}{2\sqrt{x}} (f(\sqrt{x}) + f(-\sqrt{x}))
  \end{align*}
\end{itemize}

\section{stronzate varie}
\subsection{
  mancanza di memoria}
una va è senza memoria quando il comportamento passato di questa può essere bellamente ignorato nel determinare il suo comportamento futuro

la formula è che, (con $\alpha \geq \beta$, di solito)
\begin{align*}
  \mathbb{P}(X > \alpha | X > \beta) &= \mathbb{P}(X > (\alpha - \beta)) \\
  \mathbb{P}(X = \alpha | X > \beta) &= \mathbb{P}(X = (\alpha - \beta)) \\
  \mathbb{P}(X < \alpha | X > \beta) &= \mathbb{P}(X < (\alpha - \beta))
\end{align*}

godono di questa proprietà le distribuzioni
\begin{itemize}
  \item Esponenziale
  \item Geometrica
  \item Geometrica modificata
\end{itemize}

\section{sono solo formulette}
\begin{itemize}
\item \textbf{Valore atteso}:

  \begin{itemize}
    \item Variabili aleatorie discrete
      \[\mathbb{E}[X] = \sum_{x \in \{valori\ assumibili\ da\ X\}} x \mathbb{P}(X = x) \]
    \item Variabili aleatorie continue
      \[\mathbb{E}[X] = \int_{x \in \{valori\ assumibili\ da\ X\}} x f_X(x) dx \]
  \end{itemize}
\item \textbf{Varianza}:
  \[Var[X] = \mathbb{E}[X^2] - {(\mathbb{E}[X])}^2 \]
\end{itemize}

\section{procedure esercizi}
\begin{flushright}
  \hspace{-2cm}
  \textsc{Questa è una procedura del tutto sistematica} \\
  \textit{San Giovanni Battistelli} \\
  \vspace{0.5cm}
  \textsc{Eh, mica tanto} \\
  \textit{Rodolfo ``Rodman'' Boccia}
\end{flushright}

\subsection{urne e dadi}
\begin{align*}
  P(quella\ faccia) &\times P(estrazione\ data\ quella\ faccia) \\
  P(altra\ faccia) &\times P(estrazione\ data\ altra\ faccia)
\end{align*}
per la probabilità dell'estrazione fai
\subsubsection{con un solo colore}
\begin{itemize}
\item \textbf{Senza reimbussolamento}
  \[
  \frac{casi\ possibili}{casi\ totali} =
  \frac{\binom{n.\ quel\ colore}{n.\ estratte}}{\binom{tutta\ urna}{n.\ estratte}}
  \]
\item \textbf{Con reimbussolamento}
  \[
  {P(viene\ quella)}^{volte\ che\ estrai}
  \]
\end{itemize}

\subsubsection{con più colori}
\begin{itemize}
  \item \textbf{Senza reimbussolamento}
  \[
    \frac{casi\ possibili}{casi\ totali} =
    \frac{
      \binom{n.\ quel\ colore}{n.\ estratte\ quel\ colore}
      \binom{n.\ altro\ colore}{n.\ estratte\ altro\ colore}
      \binom{n.\ altro\ colore\ ancora}{n.\ estratte\ altro\ colore\ ancora}
      \ldots
    }
    {\binom{tutta\ urna}{n.\ estratte}}
  \]
  \item \textbf{Con reimbussolamento}
    \[
      (\prod_{i} P(viene\ colore_i)^{volte\ che\ estrai\ colore_i} )
      \times\ binomiale\ per\ far\ tornare
    \]
\end{itemize}

il binomiale è per mettere tutti gli ordini in cui possono uscirti i colori, se hai due colori, estratti $a$ e $b$ volte, allora il binomiale viene
\[\binom{a+b}{b}\]
se hai tre colori estratti $a$, $b$, e $c$ volte, allora il binomiale viene
\[\binom{a+b+c}{c} \binom{a+b}{b}\]

\subsubsection{al contrario}
fai con le condizionate, vaffanculo
\[P(testa|estrazione) = \frac{P(testa\ e\ estrazione)}{P(dell'intero\ esperimento)
  \footnote{vale a dire, la P che hai ottenuto dall'esercizio prima}}\]


\subsection{marginalizzazione}
se ho
\[P(X,Y) = <qualche\ funzione\ di\ X,Y>\]
allora
\begin{align*}
  P(X) &= \int_{y} P(X,Y) dy \\
  P(Y) &= \int_{x} P(X,Y) dx
\end{align*}

\end{document}
